\documentclass{article}
\def\platformdir{.}
%!TEX root = ../main.tex
\usepackage[utf8]{inputenc}
\usepackage{geometry}
\usepackage[english,norsk]{babel}

\usepackage{iflang}

%Math stuff
\usepackage{amsmath}
\usepackage{latexsym}
\usepackage{mathtools}
\usepackage{bm}
%%% : Used by addmargin - problemstyle
\usepackage{scrextend}

%%% : Used to hide solutions
\usepackage{environ}


%General latex stuff
\usepackage{fancyhdr}
\usepackage{graphicx}
\usepackage{color}
\usepackage{tikz}
\usepackage{float}
\usepackage{lastpage}
\usepackage{stackengine}
\usepackage{scalerel}
\usepackage{listings}
\usepackage{tabularx}
\usepackage{needspace}
\usepackage[european,cuteinductors]{circuitikz}
\usepackage{xstring}
\usepackage{pdfpages}
\usepackage{\platformdir/src/olr}
\usepackage{\platformdir/src/explain}
\usepackage{enumerate}
\usetikzlibrary{calc}

%!TEX root = ../main.tex
\lstset{
basicstyle=\ttfamily,
backgroundcolor=\color{gray}
}

\begin{document}
%!TEX root = ../../README.tex
\definecolor{lightgrey}{rgb}{0.9,0.9,0.9}
\definecolor{darkgreen}{rgb}{0,0.6,0}

\lstset{language=[LaTeX]TeX,
texcsstyle=*\bf\color{blue},
numbers=none,
breaklines=true,
keywordstyle=\color{darkgreen},
commentstyle=\color{red},
frame=none,
tabsize=2,
backgroundcolor=\color{lightgrey}
% otherkeywords={\bac}
}
% \lstset{language=[LaTeX]TeX,
% texcsstyle=*\bf\color{blue},
% numbers=none,
% breaklines=true,
% keywordstyle=\color{darkgreen},
% commentstyle=\color{red},
% otherkeywords={$, \{, \}, \[, \]},
% frame=none,
% tabsize=2,
% backgroundcolor=\color{lightgrey},
% caption=LaTeX example
% }

\lstset{language=[LaTeX]TeX}
\noindent Dette er et latex-oppsett designet for å lage oppgaver med løsningsforslag. Dette er implementert med PDFLaTeX via TexLive/Latexmk i Atom. Anbefales å bruke Atom eller Sublime Text for å bygge settet. Et IDE, som for eksempel TeXStudio, kan by på utfordringer.\\

\section*{Quickstart}
Det ligger en eksempelfil \lstinline{sett/example.tex}. For å bygge den gå i \lstinline{config.tex} og legg inn
\lstinline|\Sets{example}| og bygg \lstinline{plattform.tex}.\\
\subsection*{Ryddighet}
Lag en mappestruktur som ser slik ut:
\begin{verbatim}
	Mitt prosjekt/
		- Oppgaveplattform/ <- (denne mappa)
		- sett/
		- main.tex
\end{verbatim}
i main.tex skriver du følgende kode:
\begin{lstlisting}
	\def\platformdir{Oppgaveplattform}
	%%%%%% Configure in config.tex %%%%%%%%
\documentclass{article}
%!TEX root = ../main.tex
\makeatletter
\@ifundefined{hookdir}{
    \def\hookdir{local}
}{}
\makeatother

%!TEX root = ../main.tex
\makeatletter
\newcommand{\betterif}[1]{
\count@ \escapechar \escapechar \m@ne \let #1\iffalse \@if #1\iftrue \@if #1\iffalse \escapechar \count@
% \count@ \escapechar \escapechar \m@ne \let #1\iftrue \escapechar \count@
}
% \testr\abc
% \yo
\makeatother


%!TEX root = ../main.tex
\usepackage[utf8]{inputenc}
\usepackage{geometry}
\usepackage[english,norsk]{babel}

\usepackage{iflang}

%Math stuff
\usepackage{amsmath}
\usepackage{latexsym}
\usepackage{mathtools}
\usepackage{bm}
%%% : Used by addmargin - problemstyle
\usepackage{scrextend}

%%% : Used to hide solutions
\usepackage{environ}


%General latex stuff
\usepackage{fancyhdr}
\usepackage{graphicx}
\usepackage{color}
\usepackage{tikz}
\usepackage{float}
\usepackage{lastpage}
\usepackage{stackengine}
\usepackage{scalerel}
\usepackage{listings}
\usepackage{tabularx}
\usepackage{needspace}
\usepackage[european,cuteinductors]{circuitikz}
\usepackage{xstring}
\usepackage{pdfpages}
\usepackage{\platformdir/src/olr}
\usepackage{\platformdir/src/explain}
\usepackage{enumerate}
\usetikzlibrary{calc}

\inputx{\hookdir/packages}

%!TEX root = ../main.tex
\newcommand{\figref}[1]{\figurename~\ref{#1}}

\inputx{\hookdir/commands}

%!TEX root = ../main.tex
\newcommand{\pTitleLeading}{1em}
\newcommand{\ppLeading}{0.5em}
\newcommand{\pMargin}{2em}
\newcommand{\ppMargin}{1em}
\newcommand{\vsSize}{1em}
\newcommand{\vs}{\vspace{\vsSize}}
\makeatletter
\newcommand\refcounter[1]{
\edef\@currentlabel{#1}%
}
\makeatother
\newcounter{ProblemCounter}
\newcounter{PartProblemCounter}[ProblemCounter]
\renewcommand{\thePartProblemCounter}{\alph{PartProblemCounter}}

\newcounter{CyclicProblemIndexCounter}
%% Reference problems
\DeclareRobustCommand{\pplabel}[1]{
	\refcounter{\theProblemCounter}\label{pr:\setNo:#1}
	\refcounter{\thePartProblemCounter}\label{pp:\setNo:#1}
}
\let\pptag = \pplabel

\newcommand\ppref[1]{
	\ref{pr:\setNo:#1}\ref{pp:\setNo:#1}\relax%
}
\newcommand\pref[1]{
	\ref{pr:\setNo:#1}\relax%
}



%% Define headers used throughout, mostly for language support
\def\ProblemHeader{Problem}
\def\SolutionHeader{Solution}
\def\DiscussionHeader{Discussion}
\def\IdeaHeader{Idea}
\IfLanguageName{norsk}{
	\def\ProblemHeader{Oppgave}
	\def\SolutionHeader{Løsningsforslag}
	\def\DiscussionHeader{Diskusjonsmomenter}
	\def\IdeaHeader{Idé}
}{}


%% %% %% %% %% %% %% %% %% %% %% %% %% %% %%
%%     Definition of problem commands     %%
%% %% %% %% %% %% %% %% %% %% %% %% %% %% %%


%%% Style for problems and part problems
%% bp: begin problem, ep: end proble

%% Begin custom problem, that is give it a custom name
\newcommand{\bcp}[1]{
	\Needspace{7\baselineskip} % Won't put it on the page unless there is room
	{\large{\textbf{#1}}}
	\vspace{\pTitleLeading}
	\normalsize
	\begin{addmargin}{\pMargin}
}

%% Old definition of begin problem for compatibility
\newcommand{\bp}[1]{
	\bcp{\ProblemHeader\ #1}
}

%% Begin automagically numbered problem
\newcommand{\bnp}{
	\stepcounter{ProblemCounter}
	\bcp{\ProblemHeader\ \theProblemCounter}
}
%% Create alias for compability
\let\bpa = \bnp

%% Begin mandatory problem
%% Uses /ProblemHeader and \ProblemCounter
%% and adds a \star to mark a problem as mandatory
\newcommand{\bmp}[1]{
	\stepcounter{ProblemCounter}
	\bcp{\ProblemHeader\ \theProblemCounter $(\star)$}
}

%% Ends problem
%% Commonly used for all commands to begin one
\newcommand{\ep}{
	\end{addmargin}
	\vspace{2em}
}
\let\epa = \ep %Because I keep writing it // Convenience alias


%% bpp: begin part problem. Argument is part problem name
%% epp: end part problem
\newcommand{\bpp}[1]{
	\Needspace{3\baselineskip} % Won't put it on the page unless there is room
	\vspace{\ppLeading}
	   \begin{addmargin}{\ppMargin}
		\hspace{-5ex}\large{\textbf{(#1)}}\normalsize
		}
\newcommand{\epp}{\hfill
\end{addmargin}
}



%% Automatic problem numbering
\newcommand{\bpa}[0]{
\stepcounter{problem}
\bp{\theproblem}
}

\newif\ifLabelBranches
\let\LabelBranches = \LabelBranchestrue

%% Automatic problem numbering with branching
%% Takes a list of {path1,...,pathN}
%% Outcome is governed by \BranchNos, set by \Branches{},
%% which indexes the argument list. If \BranchNos is empty,
%% all indexes are printed. If \BranchNos is not empty but
%% does not contain any relevant index for the input list,
%% nothing is printed.
\newcommand{\branchProblem}[1]{
	\stepcounter{problem}
	\newcounter{BranchIndexCounter}

	\ifx\branchNos\empty
		\foreach \path in #1{
			\stepcounter{BranchIndexCounter}
			\bp{\theproblem\ - Branch \Alph{BranchIndexCounter}}
			\input{\path}
			\ep
		}
	\else
		\foreach \path in #1{
			\stepcounter{BranchIndexCounter}
			\foreach \branchNo in \branchNos{
				\ifnumcomp{\branchNo}{=}{\theBranchIndexCounter}{
					\ifLabelBranches
						\bp{\theproblem\ - Branch \Alph{BranchIndexCounter}}
					\else
						\bp{\theproblem}
					\fi
					\input{\path}
					\ep
				}{
					%% EMPTY ELSE STATEMENT
				}
			}
		}
	\fi
}


%% %% %% %% %% %% %% %% %% %% %% %% %% %% %%
%%   When to show different environments  %%
%% %% %% %% %% %% %% %% %% %% %% %% %% %% %%


%% Showing and hiding environments
%% Hidden by default. Show with, preferably in main
%\displaySolutions
%\displayDirections
%\displayIdeas

\def\pSkip{\stepcounter{ProblemCounter}}
\def\ppSkip{\stepcounter{PartProblemCounter}}

\def\bex{}
\def\eex{}
% \def\bex{\ifDisplaySolutions\textbf{Oppgave}:\\\fi}
% \def\eex{}

\newif\ifDisplaySolutions
\newif\ifDisplayDirections
\newif\ifDisplayIdeas

\let\DisplaySolutions = \DisplaySolutionstrue
\let\DisplayDirections = \DisplayDirectionstrue
\let\HideDirections = \DisplayDirectionsfalse
\let\DisplayIdeas = \DisplayIdeastrue

\newcommand\IsTeacherCopy[0]{
	\DisplaySolutions
	\DisplayDirections
}

\newcommand\SolutionManual[0]{
	\DisplaySolutions
	\HideDirections
}

\NewEnviron{solution}{\ifDisplaySolutions\paragraph{\SolutionHeader:}\BODY\else\fi}
\NewEnviron{direction}{\ifDisplayDirections\paragraph{\DiscussionHeader:}\BODY\else\fi}
\NewEnviron{idea}{\ifDisplayIdeas\paragraph{\IdeaHeader:}\BODY\fi}


%%Clearpage if solution included
\def\solbreak{\ifDisplayDirections\else\ifDisplaySolutions\clearpage\fi\fi}
\def\probbreak{\ifDisplaySolutions\else\ifDisplayDirections\else\clearpage\fi\fi}
\def\tbreak{\ifDisplaySolutions \ifDisplayDirections \clearpage \fi \fi}
\def\tosbreak{\ifDisplayDirections\clearpage\fi\ifDisplaySolutions\clearpage\fi}
\let\sbreak = \solbreak
\let\pbreak = \probbreak
\newif\ifshowtoc
\newif\ifGenerateTOC
\let\GenerateTOC = \GenerateTOCtrue
\let\ShowTOC = \showtoctrue

\newif\ifResetProblemCounterOnSet
\let\ResetProblemCounterOnSet = \ResetProblemCounterOnSettrue
\let\KeepProblemCounterOnSet = \ResetProblemCounterOnSetfalse

%!TEX root = ../main.tex
%%% Definitions and stuff
\def\versionText{}
\def\setNo{}
\def\setNos{}
\def\courseCode{REA-0100}
\def\courseTitle{Les README.pdf}
\def\exerciseSetName{Oppgavesett}
\newif\ifBuildIntent
\newif\ifBuildOverview
\newif\ifBuildReadme
\newif\ifHideFootLine

\newif\ifBlankBeforeNew

\let\BlankBeforeNew = \BlankBeforeNewtrue
\let\BuildOverview = \BuildOverviewtrue
\let\BuildReadme = \BuildReadmetrue
\let\HideFootLine = \HideFootLinetrue

\newcommand{\Fagkode}[1]{\def\courseCode{#1}}
\newcommand{\Fagtittel}[1]{\def\courseTitle{#1}}
\newcommand{\Oppgavenavn}[1]{\def\exerciseSetName{#1}}
\newcommand{\Sets}[1]{\def\setNos{#1}}

%% Headers and footers
\pagestyle{fancy}
\fancyhf{}

\chead{\courseCode\ \courseTitle}
\newif\ifCustomBuild
\CustomBuildtrue
\let\buildAll = \CustomBuildfalse


\rfoot{Side \thepage\ av \pageref{LastPage}}
\cfoot{}
\AtBeginDocument{
\ifHideFootLine
\else
  \renewcommand{\footrulewidth}{0.4pt}
\fi
%%% Dynamic header and footer based on ifs in main.tex
\ifDisplaySolutions
  \ifDisplayDirections
    %% If display solutions AND direction
    \def\versionText{Løsnings- og diskusjonsforslag}
    \else
    %% If (display solutions) AND (NOT display direction)
    \def\versionText{Løsningsforslag}
  \fi
\else
  \ifDisplayDirections
    %if (display direction) AND (NOT display solution)
    \def\versionText{Diskusjonsforslag}
  \else
    %If (NOT display direction) AND (NOT display solution)
    \def\versionText{}
  \fi
\fi
\rhead\versionText
\ifDisplayIdeas
  \lfoot{\color{red}Oppgaveideer er inkludert}
\fi
\lhead{\exerciseSetName\ \setNo}
\lfoot{\currentSubject}
}

%%% Front page %%%
\setlength{\parindent}{0pt}

\betterif\ifDisplayTitle
\let\DisplayTitle = \DisplayTitletrue
\let\HideTitle = \DisplayTitlefalse

\geometry{paper=a4paper, bottom=3cm, top=3cm, footnotesep=3cm}

\inputhook{pagestyle}
%!TEX root = ../main.tex
\lstset{
basicstyle=\ttfamily,
backgroundcolor=\color{gray}
}

%!TEX root = ../main.tex
\DisplayTitle
\ResetProblemCounterOnSet
\def\facultyname{Fakultet for naturvitenskap og teknologi}
\def\bibStyle{unsrt}

%!TEX root = ../main.tex
\newcommand{\uitTitle}{
  \IfFileExists{\hookdir/titles/\frontpage.tex}{
    \input{\hookdir/titles/\frontpage}
  }{
    \input{\platformdir/titles/\frontpage}
  }
}
\def\currentSubject{}
\newcommand\subject[1]{
  \addcontentsline{toc}{section}{#1}
  \def\currentSubject{#1}
  {\vspace{-1em}\large\begin{center}#1\end{center}}
}


\def\ProblemSetDirectory{sett/}
\newcommand{\SetProblemSetPath}[1]{\def\ProblemSetDirectory{#1/}}
\makeatletter
\newcommand\Setlang[1]{
\selectlanguage{#1}
\select@language{#1}
\def\bbl@main@language{#1}
}
\makeatother

%%

\IfFileExists{myconfig.tex}{\input{myconfig}}{
  \IfFileExists{config.tex}{
    %!TEX root = main.tex
%%%%%%%%%%%%% Before configuring. Read the readme.pdf %%%%%%%%%%%%%%%%%
% \Sets{}
% \Fagkode{}
% \Fagtittel{}

% \IsTeacherCopy
% \SolutionManual

  }{
    %!TEX root = main.tex
%%%%%%%%%%%%% Before configuring. Read the readme.pdf %%%%%%%%%%%%%%%%%
% \Sets{}
% \Fagkode{}
% \Fagtittel{}

% \IsTeacherCopy
% \SolutionManual

  }
}
%%% Problemstyle related
\def\SolutionHeader{Løsningsforslag}
\def\DiscussionHeader{Diskusjonsmomenter}
\def\IdeaHeader{Idé}
\setstring{ProblemHeader}{Oppgave}
\setstring{setnotfound}{{\Huge\color{red}\textbf{Fil \ProblemSetDirectory\setNo.tex finnes ikke}}}
\setstring{containpages}{Inneholder \pageref{LastPage} nummererte side, inkludert forside.}

\inputx{\platformdir/languages/\languagename}
%!TEX root = ../main.tex
\newcommand{\predocument}[0]{
\ifshowtoc
\ifGenerateTOC
  \tableofcontents
\else
  \makeatletter
    {\Large\contentsname\par}
    \input{\reldir/lib/generated.toc}
  \makeatother
\fi
  \thispagestyle{empty}
  \addtocounter{page}{-1}
  \clearpage
\fi
}
\newcommand{\buildset}[0]{
  \IfFileExists{sett/\setNo}{
    \ifResetProblemCounterOnSet
    \setcounter{problem}{0}
    \fi
    %% uitTitle defined in pagestyle
    \uitTitle
    \input{sett/\setNo}
    \clearpage
  }{
    \thispagestyle{empty}
    \begin{center}
      {\Huge\color{red}\textbf{Fil sett/\setNo.tex finnes ikke}}
    \end{center}
    \addtocounter{page}{-1}
    \clearpage
  }

}
\newcommand{\readTheReadme}[0]{
  \thispagestyle{empty}
  % \begin{center}
    % {\Huge Les \lstinline{README.pdf}}
  \ifBuildReadme\else
  {\Large Du må lese README.pdf. Her er den (Neste sidene)}
  \fi
  \includepdf[pages=-]{README.pdf}
  % \end{center}
}
\begin{document}
  \ifBuildReadme\readTheReadme\fi
  \ifx\setNos\empty
    \ifBuildReadme\else\readTheReadme\fi
  \else
    \foreach \setNo [count=\scount] in \setNos{
      \buildset
    }
\fi

\end{document}


\end{lstlisting}


\section*{Konfigurering}
I \lstinline|config.tex| er det lagt opp til å kunne kjøre kommandoer som bygger forskjellige utgaver av oppgavesettene.
Dette er kommandoene som er tilgjengelige:\\[2em]
\begin{tabularx}{\textwidth}{lX}
	\lstinline{\\IsTeacherCopy}&
	Legger ved løsning- og diskusjonsforslag i tillegg til oppgavesettene. Om denne ikke kjøres, vil kun oppgaver vises i oppgavesettet, og ingen løsningsforslag\\
	\lstinline{\\SolutionManual}&Legger ved løsningsforslag i tillegg til oppgavesettene. Altså ingen diskusjon.\\
	\lstinline{\\Sets}&Tar en kommaseparert liste med tall. Tallene representerer hvilke sett som skal bygges i denne PDF-en. Den kan for eksempel være \lstinline|\\Sets\{1,...,14\}| for å bygge settene 1 til og med 14.\\
	\lstinline{\\KeepProblemCounterOnSet}&Om denne er gitt, vil oppgavenummerene ikke starte på nytt for hvert sett som er gitt i \lstinline{\\Sets}. \\
	\lstinline{\\HideTitle}&Om denne er gitt vil det ikke være tittel på noen av oppgavesettene\\
	\lstinline{\\DisplayIdeas}&Default ikke gitt. Når disse oppgavesettene ble produsert, var det noen oppgaver som ble laget som utkast, men som ikke ble med i det ferdige. Du kan vise disse utkastene ved å kjøre \lstinline{\\DisplayIdeas} i config.tex.\\
	\lstinline{\\Fagkode}&Sett fagkode som vises på tittelsiden(e) og i topp- eller bunntekst. F.eks. `FYS-1002'\\
	\lstinline{\\Fagtittel}&Tittel på faget. For eksempel `Elektromagnetisme'.\\
	\lstinline{\\Oppgavenavn}&Hva slags oppgaver er det? For eksempel kan her være `Konsepptoppgavesett'. Default er `Oppgavesett'\\
	\lstinline{\\subject}&Denne er til bruk i de forskjellige settene. Oppgavesettene skal legges i `sett/'-mappen.
												Subject tar et argument som skal inneholde temaet til det nåværende settet, og legges da automatisk i bunnekst og innholdsfortegnelse.
\end{tabularx}
I tillegg til \lstinline{config.tex} kan en lage \lstinline{myconfig.tex} som overstyrer. Dette for beleilighet ved samarbeid i et SVM som for eksempel git.\\
\textbf{OBS!} Tittel vises ikke i filer som ikke har et tall som filnavn!
\clearpage
\section*{Kommandoer og environments}
Kommandoer, utenom de som er definert i \lstinline{\platformdir/commands.tex}, er følgende spesielle kommandoer definert\\
\begin{tabularx}{\textwidth}{lX}
	\lstinline{\\solbreak}&Setter inn ny side dersom kun løsningsforslag er inkludert (altså om \lstinline{\\SolutionManual} er gitt)\\
	\lstinline{\\tbreak}&Setter inn ny side dersom \lstinline{\\IsTeacherCopy} er gitt (om både løsningsforslag og diskusjonsforslag er med)\\
	\lstinline{\\probbreak}&Setter in ny side dersom kun oppgavesett bygges. Altså hverken \lstinline{\\IsTeacherCopy} eller \lstinline{\\SolutionManual} er gitt.\\
	\lstinline{\\bpa}&Begin Problem Automatic (Numbering). Denne setter inn tittel "Oppgave TALL", og lager marger.\\
	\lstinline{\\ep}&End problem. Denne fjerner margen fra \lstinline{\\bpa}. For hver \lstinline{\\bpa} må det eksistere en \lstinline{\\ep}\\
	\lstinline{\\bppa}&Lik som \lstinline{\\bpa}, bare at den legger inn deloppgaver (Begin Part Problem Automatic).\\\lstinline{\\epp}&End Part Problem\\
	\lstinline{\\pplabel}&Tar et argument som label. Denne labelen kan så brukes i \lstinline{\\ppref} eller \lstinline{\\pref} for å referere til delopgave eller oppgave henholdsvis.\\
	\lstinline{\\ppref}&Tar et argument som er label i fra \lstinline{\\pplabel}. Denne printer så ut oppgave og deloppgave som \lstinline{\\pplabel} med tilsvaren label var kalt i. F.eks 1a.\\
	\lstinline{\\pref}&Lik som \lstinline{\\ppref} med bare at den ikke printer deloppgave, men kun oppgavenummer. F.eks. 1.
\end{tabularx}
Environments:\\
\begin{tabularx}{\textwidth}{lX}
	\lstinline{solution} og \lstinline{direction} & Alt i dissen environmentene vises kun dersom \lstinline{\\IsTeacherCopy} er på. \lstinline{direction} er ment som en peker til hva vi har tenkt når vi lagde oppgaven, og \lstinline{solution} er vårt løsningsforslag
\end{tabularx}\section*{Lib-mappen}
"Logikken" og tex-filer ligger i \lstinline|\platformdir/|-mappen.
\lstinline|\platformdir/codestyle| bestemmer hvordan \lstinline|lstlistings| skal se ut. \lstinline|\platformdir/commands| inneholder kommandodefinisjoner, \lstinline|document| er logikken bak hvordan oppgavesett er implementert
\clearpage
\section*{Andre (egenskrevne) pakker}
\subsection*{OLR - Operator Left Right}
Bruk \lstinline|\OLR[p]{sin}| vil gjøre at LaTeX-kommandoen \lstinline{\sin} vil kunne ta argument, som blir satt i left/right parantesmodus. p er parantes, b er square brackets \lstinline|[]| og c er curly braces \lstinline|{}|.
Eksempel

\begin{lstlisting}
	\OLR[p]{sin}
	\[ \sin{\sqrt{\frac{\omega}{k}}} \]
\end{lstlisting}
gir:\\
%%% Her er det ikke brukt OLR fordi \OLR må legges i preamble %%%
\[ \sin\left(\sqrt{\frac{\omega}{k}}\right) \]

\subsection*{explain}
Lager piler hit og dit. Først må en bruke \lstinline|\tagexp{MYTAGNAME}{MATH}| i mathmode og senere bruke \lstinline|\explain[valg]{MYTAGNAME}{FORKLARING}|. Valgene er mange, men i hovedsak er det posisjon som er interessant å endre her. Posisjoner er \lstinline{above} og \lstinline{below} i kombinasjon med \lstinline{left} og \lstinline{right}. For eksempel
\begin{lstlisting}
	$\tagexp{euconst}{\mathrm{e}}^{i\pi}=-1$
	\explain[below right]{euconst}{Eulers konstant i Eulers identitet}
\end{lstlisting}
Gir følgende:\\
$\tagexp{euconst}{\mathrm{e}}^{i\pi}=-1$
\explain[below right]{euconst}{Eulers konstant i Eulers identitet}
\vspace*{3em}
\section*{Flere konfigurerings-kommandoer}\textbf{Helst ikke bruk disse for konfigurering. De er ment for bruk i bakgrunnen.}\\
\begin{tabularx}{\textwidth}{lX}
	\lstinline{\\DisplayTitle}&Gitt ved default. Om denne er gitt vil det være en tittel på alle oppgavesettene. Motpart: \lstinline{\\HideTitle}\\
	\lstinline{\\DisplaySolutions}&Denne er ment til å kun vise løsningsforslag og oppgavetekst. Altså ikke diskusjon. Tanken er at man skal kunne gjøre dette tilgjengelig for elevene om ønskelig.\\
	\lstinline{\\DisplayDirections}&Er egentlig med for å være i bakgrunnen. Har samme effekt som \lstinline{\\DisplaySolutions}, bare med \lstinline{direction} environment.\\
\end{tabularx}
\end{document}
