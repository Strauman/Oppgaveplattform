%!TEX root = ../main.tex
\newcommand{\pTitleLeading}{1em}
\newcommand{\ppLeading}{0.5em}
\newcommand{\pMargin}{2em}
\newcommand{\ppMargin}{1em}
\newcommand{\vsSize}{1em}
\newcommand{\vs}{\vspace{\vsSize}}
\makeatletter
\newcommand\refcounter[1]{
\edef\@currentlabel{#1}%
}
\makeatother
\newcounter{ProblemCounter}
\newcounter{PartProblemCounter}[ProblemCounter]
\renewcommand{\thePartProblemCounter}{\alph{PartProblemCounter}}

\newcounter{CyclicProblemIndexCounter}
%% Reference problems
\DeclareRobustCommand{\pplabel}[1]{
	\refcounter{\theProblemCounter}\label{pr:\setNo:#1}
	\refcounter{\thePartProblemCounter}\label{pp:\setNo:#1}
}
\let\pptag = \pplabel

\newcommand\ppref[1]{
	\ref{pr:\setNo:#1}\ref{pp:\setNo:#1}\relax%
}
\newcommand\pref[1]{
	\ref{pr:\setNo:#1}\relax%
}


%% %% %% %% %% %% %% %% %% %% %% %% %% %% %%
%%     Definition of problem commands     %%
%% %% %% %% %% %% %% %% %% %% %% %% %% %% %%


%%% Style for problems and part problems
%% bp: begin problem, ep: end proble

%% Begin custom problem, that is give it a custom name
\newcommand{\bcp}[1]{
	\Needspace{7\baselineskip} % Won't put it on the page unless there is room
	{\large{\textbf{#1}}}
	\vspace{\pTitleLeading}
	\normalsize
	\begin{addmargin}{\pMargin}
}

%% Old definition of begin problem for compatibility
\newcommand{\bp}[1]{
	\bcp{\ProblemHeader\ #1}
}

%% Begin automagically numbered problem
\newcommand{\bnp}{
	\stepcounter{ProblemCounter}
	\bcp{\ProblemHeader\ \theProblemCounter}
}
%% Create alias for compability
\let\bpa = \bnp

%% Begin mandatory problem
%% Uses /ProblemHeader and \ProblemCounter
%% and adds a \star to mark a problem as mandatory
\newcommand{\bmp}[1]{
	\stepcounter{ProblemCounter}
	\bcp{\ProblemHeader\ \theProblemCounter $(\star)$}
}

%% Ends problem
%% Commonly used for all commands to begin one
\newcommand{\ep}{
	\end{addmargin}
	\vspace{2em}
}
\let\epa = \ep %Because I keep writing it // Convenience alias


%% bpp: begin part problem. Argument is part problem name
%% epp: end part problem
\newcommand{\bpp}[1]{
	\Needspace{3\baselineskip} % Won't put it on the page unless there is room
	\vspace{\ppLeading}
	   \begin{addmargin}{\ppMargin}
		\hspace{-5ex}\large{\textbf{(#1)}}\normalsize
		}
\newcommand{\epp}{\hfill
\end{addmargin}
}

\newcommand{\bppa}[0]{
	\stepcounter{PartProblemCounter}
	\bpp{\alph{PartProblemCounter}}
}
\let\eppa = \epp %Because I keep writing it // Convenience alias


%% %% %% %% %% %% %% %% %% %% %% %% %% %% %%
%%  Definition of input problem commands  %%
%% %% %% %% %% %% %% %% %% %% %% %% %% %% %%


%% Takes a path relative to builded main
%% Inputs the file within a problem
%% Use part problem within inputed file
\newcommand{\inputProblem}[1]{
	\bnp
	\input{#1}
	\ep
}

\newif\ifLabelCyclicProblems
\let\LabelCyclicProblems = \LabelCyclictrue

%% Automatic problem numbering with cyclic branching
%% Takes a list of {path1,...,pathN}
%% Outcome is governed by \BranchNos, set by \Branches{},
%% which indexes the argument list. If \BranchNos is empty,
%% all indexes are printed. If \BranchNos is not empty but
%% does not contain any relevant index for the input list,
%% nothing is printed.

\newcommand{\inputCyclicProblem}[1]{
	\stepcounter{ProblemCounter}
	\setcounter{CyclicProblemIndexCounter}{0}

	\ifx\cycleNos\empty
		\foreach \path in #1{
			\stepcounter{CyclicProblemIndexCounter}
			\bcp{\ProblemHeader\ \theProblemCounter\ -
				 Cycle \Alph{CyclicProblemIndexCounter}}
			\input{\path}
			\ep
		}
	\else
		\foreach \path in #1{
			\stepcounter{CyclicProblemIndexCounter}
			\foreach \cycleNo in \cycleNos{
				\ifnumcomp{\cycleNo}{=}{\theCyclicProblemIndexCounter}{
					\ifLabelCyclicProblems
						\bcp{\ProblemHeader\ \theProblemCounter\ -
							 Cycle \Alph{CyclicProblemIndexCounter}}
					\else
						\bcp{\ProblemHeader\ \theProblemCounter}
					\fi
					\input{\path}
					\ep
				}{
					%% EMPTY ELSE STATEMENT
				}
			}
		}
	\fi
}


%% %% %% %% %% %% %% %% %% %% %% %% %% %% %%
%%   When to show different environments  %%
%% %% %% %% %% %% %% %% %% %% %% %% %% %% %%


%% Showing and hiding environments
%% Hidden by default. Show with, preferably in main
%\displaySolutions
%\displayDirections
%\displayIdeas

\def\pSkip{\stepcounter{ProblemCounter}}
\def\ppSkip{\stepcounter{PartProblemCounter}}

\def\bex{}
\def\eex{}
% \def\bex{\ifDisplaySolutions\textbf{Oppgave}:\\\fi}
% \def\eex{}

\newif\ifDisplaySolutions
\newif\ifDisplayDirections
\newif\ifDisplayIdeas

\let\DisplaySolutions = \DisplaySolutionstrue
\let\DisplayDirections = \DisplayDirectionstrue
\let\HideDirections = \DisplayDirectionsfalse
\let\DisplayIdeas = \DisplayIdeastrue

\newcommand\IsTeacherCopy[0]{
	\DisplaySolutions
	\DisplayDirections
}

\newcommand\SolutionManual[0]{
	\DisplaySolutions
	\HideDirections
}

\NewEnviron{solution}{
	\ifDisplaySolutions
		% \paragraph{\SolutionHeader:}$ $\BODY
		\paragraph{\SolutionHeader:}$ $\BODY
	\else\fi
}
\NewEnviron{direction}{
	\ifDisplayDirections
		\paragraph{\DiscussionHeader:}$ $\BODY
	\else\fi
}
\NewEnviron{idea}{
	\ifDisplayIdeas
		\paragraph{\IdeaHeader:}$ $\BODY
	\else\fi
}


%%Clearpage if solution included
\def\solbreak{\ifDisplayDirections\else\ifDisplaySolutions\clearpage\fi\fi}
\def\probbreak{\ifDisplaySolutions\else\ifDisplayDirections\else\clearpage\fi\fi}
\def\tbreak{\ifDisplaySolutions \ifDisplayDirections \clearpage \fi \fi}
\def\tosbreak{\ifDisplayDirections\clearpage\fi\ifDisplaySolutions\clearpage\fi}
\let\sbreak = \solbreak
\let\pbreak = \probbreak
\newif\ifshowtoc
\newif\ifGenerateTOC
\let\GenerateTOC = \GenerateTOCtrue
\let\ShowTOC = \showtoctrue

\newif\ifResetProblemCounterOnSet
\let\ResetProblemCounterOnSet = \ResetProblemCounterOnSettrue
\let\KeepProblemCounterOnSet = \ResetProblemCounterOnSetfalse
